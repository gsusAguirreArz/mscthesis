\begin{extraAbstract}
\addchaptertocentry{\abstractname} 
En este trabajo se implementa una arquitectura generativa adversaria junto con la distancia de Wasserstein para generar simulaciones de cascadas atmosféricas electromagnéticas. Las señales están distribuidas en un arreglo de detectores, las cuales traen información como la dirección de arribo y la energía de la partícula primaria. Primero se investiga la generación a través de arquitecturas sencillas, como la vainilla GAN y después se comparan las señales generadas con las generadas a través de la Wasserstein GAN. Al final se manipula la dimensión del espacio latente de la arquitectura WGAN, para observar la capacidad de generación de cascadas de la arquitectura. Con este método se demuestra que usar modelos generativos son una posible alternativa para la simulación rápida de cascadas atmosféricas electromagnéticas. 
\end{extraAbstract}