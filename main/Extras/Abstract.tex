\begin{extraAbstract}
\addchaptertocentry{Abstract} 
In this research, we use adversarial network architectures together with the Wasserstein distance to generate electromagnetic air showers. The training data are two dimensional projections of ground based signal patterns, which can be used to gather relevant parameters of the shower. First we investigate the simulation capacity of vanilla networks, such as, vanilla GAN and convolutional GAN, then, we compare the simulations with each other and proceed to implement the WGAN. Once the final architecture is able to produce real like simulations, we proceed to decrease the latent dimension of the generative architecture and report the simulation capacity of the model. With this methos we demonstrate that some generative models are a good alternative for fast simulation of electromagnetic air showers.
\end{extraAbstract}