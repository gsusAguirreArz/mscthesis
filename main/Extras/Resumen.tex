\begin{abstract}
\addchaptertocentry{\abstractname} 
Las simulaciones son vitales para los procesos de trabajo teóricos y experimentales de la mayoría de las colaboraciones internacionales del campo de la Física de Altas Energías. Sin embargo los simuladores que representan el estado del arte actual del campo comienzan a encontrarse con cuellos de botella en los procesos.

En este trabajo se implementan conceptos de aprendizaje profundo y cascadas atmosféricas extensas para implementar un método que permita acelerar simulaciones de interacciones de cascadas atmosféricas. Los modelos generativos son de vital importancia en este trabajo y en la rama de las altas energías, ya que permiten una alternativa a técnicas modernas de simulación.

Los resultados experimentales muestran que el uso de modelos generativos, en específico de redes generativas adversarias funcionan como una alternativa para acelerar simuladores actuales. Las señales generadas por la red son prácticamente indistinguibles de  aquellas generadas por simuladores monte carlo tradicionales. 
\end{abstract}
% \let\cleardoublepage\clearpage
