% Chapter Template

\chapter{Metodo experimental} % Main chapter title

\label{Chapter4} % Change X to a consecutive number; for referencing this chapter elsewhere, use \ref{ChapterX}

%----------------------------------------------------------------------------------------
%	SECTION 1
%----------------------------------------------------------------------------------------
En este capítulo se detallan las técnicas utilizadas así como el método experimental para la generación y evaluación de cascadas atmosféricas.. 

\section{Recolección y manejo de datos}

Dado que el objetivo de este trabajo es implementar un modelo generativo que sea capaz generar simulaciones de cascadas atmosféricas, se necesita un banco de datos de dichas simulaciones. Una de las primeras maneras de obtener este conjunto de datos será tomar el simulador CORSIKA y generar el número de eventos necesarios para poder entrenar a la red neuronal. Como segunda opción de recolección se buscará un banco de simulaciones disponibles al público con características fundamentales como la energía de la primaria, la altura de máximo desarrollo, etc.

Como se ha mencionado en capítulos anteriores, las tareas que cada simulador conlleva para generar una cascada, incluyen subrutinas se encargan de la parte hadrónica, la parte electromagnética, el modelo de interacción, el rastreo de parámetros y procesamientos adicionales para reducir la carga computacional; por lo cual los datos que se utilizaran no deben de concentrarse en alguna de estas subrutinas para poder acotar el dominio de la simulación. 

%El análisis de simulaciones de cascadas se hace con procesamientos para obtener las distribuciones longitudinales de la cascada …

Para generar un resultado significativo sin pérdida de generalidad, se tomarán en cuenta las siguientes consideraciones:

Utilizar un número relativamente “grande” de simulaciones.
El número de observaciones debe ser un número “grande”, como referencia se intentará obtener un conjunto de datos con al menos el mismo número de observaciones que el conjunto MNIST.  

Preprocesar el conjunto de datos.
Para datos como los tiempos de arribo, el dominio de cada simulación diferirá en magnitudes que deben de tomarse en cuenta, por lo tanto técnicas de estandarización podrán ser utilizadas para tener una mejor estabilidad al entrenar el modelo. 

Separación en conjuntos de entrenamiento, validación y evaluación. 
Una vez que se cumpla lo anterior, los datos estarán listos para poder entrenar al modelo.

\section{Desarrollo del método}

A continuación se muestra el método desglosado. 
Obtener un banco de datos de simulaciones de cascadas atmosféricas acotadas a algún proceso seleccionado.

Hacer el preprocesamiento de datos adecuado para normalizar la entrada de datos y así estabilizar el entrenamiento sin dejar de tomar en cuenta las consideraciones teóricas que dicho procesamiento implica.

Elegir una arquitectura generativa adversaria y entrenarla para intentar generar nuevas cascadas.

Se tomarán las nuevas muestras y se inspeccionarán visualmente comparándolas con las observaciones originales.

Después de haber confirmado la similitud con las observaciones originales, se hará un análisis exploratorio de la dimensionalidad del vector latente y un análisis al proceso determinista de entradas y salidas de dicha red. 

Una vez logrado lo anterior, cambiar de arquitectura y repetir el proceso, si se puede.

\subsection{Mapa del método experimental}