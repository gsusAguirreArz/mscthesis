% Chapter Template

\chapter{Conlusiones} % Main chapter title

\label{Chapter6} % Change X to a consecutive number; for referencing this chapter elsewhere, use \ref{ChapterX}

%----------------------------------------------------------------------------------------
%	SECTION 1
%----------------------------------------------------------------------------------------

En este capítulo se presentan las conclusiones de este trabajo.

\section{Conclusiones puntuales obtenidas}

\begin{enumerate}
    \item Se logró implementar una arquitectura que genera cascadas electromagnéticas dado un escalar asociado a la cascada.
    \item Se mostró que con la reducción de la dimensionalidad del espacio latente se siguieron obteniendo resultados confiables,a costa de obtener colapsos de modos en entrenamientos previos.
    % \item Por último, se mostró que el promedio de vectores latentes característicos tenía “tal” efecto en las cascadas generadas.
\end{enumerate}

\section{Trabajos futuros}

Algunos de los trabajos futuros que pueden derivar directamente son:
\begin{enumerate}[label=\alph*)]
    \item Probar otros modelos generativos como InfoGAN, VAEGAN, BETA-VAE, modelos de difusión, etc. Para así hacer una comparación de cual es el modelo más efectivo para generar cascadas atmosféricas.
    \item Desenredar los parámetros latentes del modelo para así obtener un control directo de parámetros asociados a la cascada y así generar observaciones confiables.
\end{enumerate}