% Chapter Template

\chapter{Resultados} % Main chapter title

\label{Chapter5} % Change X to a consecutive number; for referencing this chapter elsewhere, use \ref{ChapterX}

%----------------------------------------------------------------------------------------
%	SECTION 1
%----------------------------------------------------------------------------------------

Este capítulo muestra algunos de los resultados obtenidos por el método experimental presentado en el capítulo anterior.

\section{Discusión de resultados obtenidos}

A continuación se describirán los resultados por cada etapa del método experimental.

\subsection*{Obtención de datos y procesamiento}

Para obtener el conjunto de datos, primero se optó por intentar generar una base de datos usando el simulador CORSIKA, sin embargo, se observo que la generación de un dataset de al menos $60000$ observaciones, tomaría un tiempo inviable con el equipo de cómputo con el que se disponía. Del mismo modo, la complejidad adicional que la interfaz del programa y la inexperiencia con el mismo, agregaba una dificultad extra a la hora de configurar los parámetros del evento. 

Por consiguiente, se decidió buscar una base de datos de simulaciones que estuvieran disponibles al público. De modo que el banco de datos que se usó fue tomado de \parencite{Erdmann2018b}, el cual contiene $100000$ simulaciones de cascadas electromagnéticas con los siguientes parámetros.

%insertar tabla de df.info()

Para la selección y preprocesamiento de datos se seleccionó la señal que contiene los tiempos de arribo y la energía asociada a cada cascada. Después se hizo el ajuste propuesto en \parencite{Erdmann2018b} a los tiempos de arribo \ref{eq:}, sin embargo el ajuste no lograba estandarizar los datos, así que se hizo un mapeo lineal al dominio $[-1,1]$.

%insertar ajuste propuesto
%insertar imagenes post procesamiento y pre procesamiento

Para el escalar de la energía de cada cascada, se decidió categorizar la variable continua en $4$ rangos energéticos, los cuales son $x$.

%insertar tabla de post procesamiento y pre procesamiento

\subsection*{Modelo y entrenamientos}

Para la arquitectura del modelo se optaron por capas convolucionales para preservar la información entre detectores y los modelos tienen las siguientes propiedades.

%insertar info de los modelos en un tabla
El único parámetro que se modificó fue la dimensión del vector latente, y se guardaron los modelos resultantes para cada dimensión, sin embargo se observó que para dimensiones bajas el modelo llegaba a un colapso de modos con más frecuencia, como se muestra a continuación. 

%insertar tabla de modelos dimension vector latente y mode collapse
%insertar gráficas de pérdidas en el entrenamiento.

\subsection*{Generación de cascadas electromagnéticas}

Cascadas electromagnéticas generadas con los modelos 
%\ref{tabla}

%insertar cascadas generadas.

Comparación con las cascadas de entrenamiento.

Comparación con $50$ cascadas generadas usando CORSIKA.

\subsection*{Combinaciones lineales de vectores latentes}

Resultados al promediar vectores latentes y hacer una combinación lineal de estos para después ser insertados en categorías energéticas diferentes.

%insertar imágenes.

\begin{table}
    \caption{The effects of treatments X and Y on the four groups studied.}
    \label{tab:treatments}
    \centering
    \begin{tabular}{l l l}
        \toprule
        \textbf{Groups} & \textbf{Treatment X} & \textbf{Treatment Y} \\
        \midrule
        1 & 0.2 & 0.8\\
        2 & 0.17 & 0.7\\
        3 & 0.24 & 0.75\\
        4 & 0.68 & 0.3\\
        \bottomrule\\
    \end{tabular}
\end{table}

(e.g. Table~\ref{tab:treatments}).


\lstinputlisting[language=Python]{Code/dfa.py}
