\documentclass[letterpaper,12pt,oneside]{book}
%\usepackage[a4paper,includeall,bindingoffset=0cm,margin=2cm,marginparsep=0cm,marginparwidth=0cm]{geometry}
\usepackage[top=1in, left=0.9in, right=1.25in, bottom=1in]{geometry}
\usepackage{bachelorstitlepageUNAM}
\usepackage[utf8]{inputenc}
%%%%%%%%%%%%%%%%%%%%%%%%%%%%%
% Comparto una plantilla para la PORTADA que us\'e en mi t\'esis
% basada en el dise\~no gen\'erico que se usa en la Facultad de Ciencias
% Para usarlo \'unicamente aseg\'urate de tener la l\'inea
% \usepackage{bachelorstitlepageUNAM} y el archivo bachelorstitlepageUNAM.sty en el mismo directorio de trabajo.
% y los campos (sin signo %) :
%\author{Nombre del Alumno}
%\title{T\'itulo de la tesis}
%\faculty{Facultad}
%\degree{Grado que obtienes}
%\supervisor{ Tutor}
%\cityandyear{Ciudad y anio}
%\logouni{nombredelescudodelaunamsinespacios}
%\logofac{NombreDeLaImagenDelEscudodeTuFacultadSinEspacios}
% Para sugerencias y comentarios: DM en twitter.com/sglvgdor
% Subir\'e mas adelante la plantilla para maestr\'ia
%%%%%%%%%%%%%%%%%%%%%%%%%%%%%

%\author{Irving Yosafat Angel Camacho}
%\title{Métodos Numéricos de la Hidrodinámica Relativista aplicados a problemas de acreción y eyección en %jets astrofísicos}
%\faculty{Escuela Nacional de Estudios Superiores\\
%            Unidad Morelia}
%\degree{Licenciado en Geociencias}
%\supervisor{Dr. Sergio Mendoza Ramos \\ 
%Dr. Sinhué A. R. Haro Corzo}
%\cityandyear{Morelia, Michoacán, 2019}
%\logouni{Escudo-UNAM}
%\logofac{logo-enes}
%
%-------------------------------





%-----------------------__--------

\usepackage[T1]{fontenc}
\usepackage[utf8]{inputenc}
\usepackage[spanish,es-nodecimaldot,es-tabla]{babel}
\usepackage{graphicx}
\usepackage{tikz} 
\usepackage{tocloft}
\graphicspath{{./figs/}}
\usepackage{setspace}

%\usepackage[round]{natbib}

\renewcommand\cftsecpresnum{\S}
\renewcommand\cftsubsecpresnum{\S}   


\begin{document}
%------------------------------

    \begin{titlepage}
        \thispagestyle{empty}
        \begin{minipage}[c][0.17\textheight][c]{0.25\textwidth}
            \begin{center}
                \includegraphics[width=3.5cm, height=3.5cm]{Escudo-UNAM.pdf}
            \end{center}
        \end{minipage}
        \begin{minipage}[c][0.195\textheight][t]{0.75\textwidth}
            \begin{center}
                \vspace{0.3cm}
                \textsc{\large Universidad Nacional Aut\'onoma de M\'exico}\\[0.5cm]
                \vspace{0.3cm}
                \hrule height2.5pt
                \vspace{.2cm}
                \hrule height1pt
                \vspace{.8cm}
                \textsc{FACULTAD DE INGENIERÍA\\
DIVISIÓN DE INGENIERÍA ELÉCTRICA }\\[0.5cm] %
            \end{center}
        \end{minipage}

        \begin{minipage}[c][0.81\textheight][t]{0.25\textwidth}
            \vspace*{5mm}
            \begin{center}
                \hskip2.0mm
                \vrule width1pt height13cm 
                \vspace{5mm}
                \hskip2pt
                \vrule width2.5pt height13cm
                \hskip2mm
                \vrule width1pt height13cm \\
                \vspace{5mm}
                \includegraphics[height=4.0cm]{logo-enes.png}
            \end{center}
        \end{minipage}
        \begin{minipage}[c][0.81\textheight][t]{0.75\textwidth}
            \begin{center}
                \vspace{1cm}

                {\large\scshape Simulación de un disparo de turbina en un reactor ESBWR por criticidad}\\[.2in]

                \vspace{2cm}            

                \textsc{\LARGE T\hspace{1.5cm}E\hspace{1.5cm}S\hspace{1.5cm}I\hspace{1.5cm}S}\\[0.5cm]
                \textsc{\large que para obtener el t\'itulo de:}\\[0.5cm]
                \textsc{\large Ingeniero en Energía}\\[0.5cm]
                \textsc{\large presenta:}\\[0.5cm]
                \textsc{\large {Juan Andrés Aguilar Huesca}}\\[2cm]          

                \vspace{0.5cm}

                {\large\scshape Tutores:\\[0.3cm] {Mtro. Ulises  Adair Hernandez Hurtado \\ 
Mtro. Edgar Slazar Salazar}}\\[.2in]

                \vspace{0.5cm}

                \large{Ciudad Universitaria, CD.MX}{ }{2020}
            \end{center}
        \end{minipage}
    \end{titlepage}



%---------------------------------
\frontmatter
%\maketitle
\chapter*{}
\begin{flushright}%
  \emph{Dedicatoria ...}
  \thispagestyle{empty}
\end{flushright}

\chapter{Agradecimientos}
\spacing{1.5}%\doublespacing

\chapter{Notación}

\chapter{Introducción}

\tableofcontents
\listoffigures

    
\mainmatter

\chapter{Hidrodinámica relativista} \section{Introducción}
    \section{Tensor energía-momento}
    \section{Ecuaciones de la hidrodinámica relativista}
    \section{Ecuación de la conservación de la entropía}
    \section{Ondas de choque relativistas}

\chapter{Métodos numéricos}

\chapter{Código aztekas}
    \section{Antecedentes}
    \section{Método x}
        \subsection{Análisis preliminar}
        \subsection{Discretización aztekas}
        \section{Estructura del código}
        \subsection{Archivos base}
        \subsection{Achivos creados}
    \section{Funciones importantes}
    
\chapter{Pruebas numéricas}
    \section{Tubo de choque}
    \section{Casos particulares}
        \subsection{Colisión de dos ondas de choque}
        \subsection{Choques internos en jets relativistas}
    
\chapter{Conclusiones}  

%\bibliographystyle{humannat}
%\bibliography{references}

\backmatter%@sglvgdor


\end{document}